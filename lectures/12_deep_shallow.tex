\part{Deep and Shallow Embeddings}

\frame[plain]{\partpage}

\begin{frame}
\frametitle{Deep and Shallow Embeddings}

\begin{itemize}
\item often one models some kind of formal language
\item important design decision: use \emph{deep} or \emph{shallow} embedding
\item in a nutshell:
\begin{itemize}
\item shallow embeddings just model semantics
\item deep embeddings model syntax as well
\end{itemize}
\item a shallow embedding directly uses the HOL logic
\item a deep embedding
\begin{itemize}
\item defines a datatype for the syntax of the language
\item provides a function to map this syntax to a semantic
\end{itemize}
\end{itemize}

\end{frame}


\begin{frame}[fragile]
\frametitle{Example: Embedding of Propositional Logic I}

\begin{itemize}
\item propositional logic is a subset of HOL
\item a shallow embedding is therefore trivial
\end{itemize}

\begin{semiverbatim}\scriptsize
val sh_true_def    = Define `sh_true = T`;
val sh_var_def     = Define `sh_var (v:bool) = v`;
val sh_not_def     = Define `sh_not b = \holNeg{}b`;
val sh_and_def     = Define `sh_and b1 b2 = (b1 \holAnd{} b2)`;
val sh_or_def      = Define `sh_or b1 b2 = (b1 \holOr{} b2)`;
val sh_implies_def = Define `sh_implies b1 b2 = (b1 ==> b2)`;
\end{semiverbatim}

\end{frame}


\begin{frame}[fragile]
\frametitle{Example: Embedding of Propositional Logic II}

\begin{itemize}
\item we can also define a datatype for propositional logic
\item this leads to a deep embedding

\begin{semiverbatim}\scriptsize
val _ = Datatype `bvar = BVar num`
val _ = Datatype `prop = d_true | d_var bvar | d_not prop
                       | d_and prop prop | d_or prop prop
                       | d_implies prop prop`;

val _ = Datatype `var_assignment = BAssign (bvar -> bool)`
val VAR_VALUE_def = Define `VAR_VALUE (BAssign a) v = (a v)` 
                     
val PROP_SEM_def = Define `
  (PROP_SEM a d_true = T) \holAnd{}
  (PROP_SEM a (d_var v) = VAR_VALUE a v) \holAnd{}
  (PROP_SEM a (d_not p) = \holNeg{}(PROP_SEM a p)) \holAnd{}
  (PROP_SEM a (d_and p1 p2) = (PROP_SEM a p1 \holAnd{} PROP_SEM a p2)) \holAnd{}
  (PROP_SEM a (d_or p1 p2) = (PROP_SEM a p1 \holOr{} PROP_SEM a p2)) \holAnd{}
  (PROP_SEM a (d_implies p1 p2) = (PROP_SEM a p1 ==> PROP_SEM a p2))`
\end{semiverbatim}
\end{itemize}
\end{frame}

\begin{frame}
\frametitle{Shallow vs.\ Deep Embeddings}

\newcommand{\dummyitem}{\item[] \leavevmode\phantom{gg}}

\begin{minipage}[t]{.46\textwidth}
\begin{block}{Shallow}
\begin{itemize}
\item quick and easy to build
\item extensions are simple
\end{itemize}
\end{block}
\end{minipage}\qquad
\begin{minipage}[t]{.46\textwidth}
\begin{block}{Deep}
\begin{itemize}
\item can reason about syntax
\item allows verified implementations
\item sometimes tricky to define 
\begin{itemize}
\item \eg bound variables
\end{itemize}
\end{itemize}
\end{block}
\end{minipage}
\bigskip

\begin{block}{Important Questions for Deciding}
\begin{itemize}
\item Do I need to reason about syntax?
\item Do I have hard to define syntax like bound variables?
\item How much time do I have?
\end{itemize}
\end{block}
\end{frame}

\begin{frame}[fragile]
\frametitle{Example: Embedding of Propositional Logic III}

\begin{itemize}
\item with deep embedding one can easily formalise syntactic properties like
\begin{itemize}
\item Which variables does a propositional formula contain?
\item Is a formula in negation-normal-form (NNF)?
\end{itemize}
\item with shallow embeddings
\begin{itemize}
\item syntactic concepts can't be defined in HOL
\item however, they can be defined in SML
\item no proofs about them possible
\end{itemize}
\end{itemize}

\begin{semiverbatim}\scriptsize
val _ = Define `
  (IS_NNF (d_not d_true) = T) \holAnd{} (IS_NNF (d_not (d_var v)) = T) \holAnd{}
  (IS_NNF (d_not _) = F) \holAnd{}\medskip
  (IS_NNF d_true = T) \holAnd{} (IS_NNF (d_var v) = T) \holAnd{}
  (IS_NNF (d_and p1 p2) = (IS_NNF p1 \holAnd{} IS_NNF p2)) \holAnd{}
  (IS_NNF (d_or p1 p2) = (IS_NNF p1 \holAnd{} IS_NNF p2)) \holAnd{}
  (IS_NNF (d_implies p1 p2) = (IS_NNF p1 \holAnd{} IS_NNF p2))`
\end{semiverbatim}
\end{frame}




\begin{frame}
\frametitle{Verified vs.\ Verifying Program}
\newcommand{\dummyitem}{\item[] \leavevmode\phantom{gg}}

\begin{minipage}[t]{.46\textwidth}
\begin{block}{Verified Programs}
\begin{itemize}
\item are formalised in HOL
\item their properties have been proven once and for all
\item all runs have proven properties
\item are usually less sophisticated, since they need verification
\item is what one wants ideally
\item often require deep embedding
\end{itemize}
\end{block}
\end{minipage}\qquad
\begin{minipage}[t]{.46\textwidth}
\begin{block}{Verifying Programs}
\begin{itemize}
\item are written in meta-language
\item they produce a separate proof for each run
\item only certain that current run has properties
\item allow more flexibility, \eg fancy heuristics
\item good pragmatic solution
\item shallow embedding fine
\end{itemize}
\end{block}
\end{minipage}
\end{frame}


\begin{frame}
\frametitle{Summary Deep vs.\ Shallow Embeddings}
\begin{itemize}
\item deep embeddings require more work
\item they however allow reasoning about syntax
\begin{itemize}
\item induction and case-splits possible
\item a semantic subset can be carved out syntactically
\end{itemize}
\item syntax sometimes hard to define for deep embeddings
\item combinations of deep and shallow embeddings common
\begin{itemize}
\item certain parts are deeply embedded
\item others are embedded shallowly
\end{itemize}
\end{itemize}
\end{frame}



%%% Local Variables:
%%% mode: latex
%%% TeX-master: "current"
%%% End:
