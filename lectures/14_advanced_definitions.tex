\part{Advanced Definition Principles}

\frame[plain]{\partpage}

\section{Inductive and Coinductive Relations}

\begin{frame}
\frametitle{Relations}

\begin{itemize}
\item a relation is a function from some arguments to \texttt{bool}
\item the following example types are all types of relations:
\begin{itemize}
\item \hol{:\ 'a -> 'a -> bool}
\item \hol{:\ 'a -> 'b -> bool}
\item \hol{:\ 'a -> 'b -> 'c -> 'd -> bool}
\item \hol{:\ ('a \# 'b \# 'c) -> bool}
\item \hol{:\ bool}
\item \hol{:\ 'a -> bool}
\end{itemize}
\item relations are closely related to sets
\begin{itemize}
\item \hol{R a b c <=> (a, b, c) IN \{(a, b, c) | R a b c\}}
\item \hol{(a, b, c) IN S <=> (\textbsl{}a b c.\ (a, b, c) IN S) a b c}
\end{itemize}
\end{itemize}
\end{frame}


\begin{frame}
\frametitle{Relations II}

\begin{itemize}
\item relations are often defined by a set of \emph{rules}
\begin{minipage}{.9\textwidth}
\begin{exampleblock}{Definition of Reflexive-Transitive Closure}
The transitive reflexive closure of a relation \hol{R : 'a -> 'a -> bool} can
be defined as the least relation \hol{RTC R} that satisfies the following rules:
\bigskip 
\begin{center}
$\inferrule*{\hol{R x y}}{\hol{RTC R x y}}$\quad
$\inferrule*{\ }{\hol{RTC R x x}}$\quad
$\inferrule*{\hol{RTC R x y}\\\hol{RTC R y z}}{\hol{RTC R x z}}$
\end{center}
\end{exampleblock}
\end{minipage}
\item if the rules are monoton, a least and a greatest fix point exists (Knaster-Tarski theorem)
\item least fixpoints give rise to \emph{inductive relations}
\item greatest fixpoints give rise to \emph{coinductive relations}
\end{itemize}
\end{frame}

\begin{frame}
\frametitle{(Co)inductive Relations in HOL}
\begin{itemize}
\item \ml{(Co)IndDefLib} provides infrastructure for defining (co)inductive relations
\item given a set of rules \hol{Hol\_(co)reln} defines (co)inductive relations
\item 3 theorems are returned and stored in current theory
\begin{itemize}
\item a rules theorem --- it states that the defined constant satisfies the rules
\item a cases theorem --- this is an equational form of the rules showing that the defined relation is indeed a fixpoint
\item a (co)induction theorem
\end{itemize}
\item additionally a strong (co)induction theorem is stored in current theory
\end{itemize}
\end{frame}


\begin{frame}[fragile]
\frametitle{Example: Transitive Reflexive Closure}
\begin{semiverbatim}\scriptsize
> \hol{val (RTC_REL_rules, RTC_REL_ind, RTC_REL_cases) = Hol_reln `
    (!x y.   R x y                          ==> RTC_REL R x y) \holAnd{}
    (!x.                                        RTC_REL R x x) \holAnd{}
    (!x y z. RTC_REL R x y \holAnd{} RTC_REL R y z ==> RTC_REL R x z)`}

val RTC_REL_rules = |- !R.
  (!x y. R x y ==> RTC_REL R x y) \holAnd{} (!x. RTC_REL R x x) \holAnd{}
  (!x y z. RTC_REL R x y \holAnd{} RTC_REL R y z ==> RTC_REL R x z)

val RTC_REL_cases = |- !R a0 a1.
  RTC_REL R a0 a1 <=>
  (R a0 a1 \holOr{} (a1 = a0) \holOr{} ?y. RTC_REL R a0 y \holAnd{} RTC_REL R y a1)
\end{semiverbatim}
\end{frame}


\begin{frame}[fragile]
\frametitle{Example: Transitive Reflexive Closure II}
\begin{semiverbatim}\scriptsize
val RTC_REL_ind = |- !R RTC_REL'.
  ((!x y. R x y ==> RTC_REL' x y) \holAnd{} (!x. RTC_REL' x x) \holAnd{}
   (!x y z. RTC_REL' x y \holAnd{} RTC_REL' y z ==> RTC_REL' x z)) ==>
  (!a0 a1. RTC_REL R a0 a1 ==> RTC_REL' a0 a1)


> \hol{val RTC_REL_strongind = DB.fetch "-" "RTC_REL_strongind"}

val RTC_REL_strongind = |- !R RTC_REL'.
  (!x y. R x y ==> RTC_REL' x y) \holAnd{} (!x. RTC_REL' x x) \holAnd{}
  (!x y z.
      RTC_REL R x y \holAnd{} RTC_REL' x y \holAnd{} RTC_REL R y z \holAnd{}
      RTC_REL' y z ==>
      RTC_REL' x z) ==>
  ( !a0 a1. RTC_REL R a0 a1 ==> RTC_REL' a0 a1)
\end{semiverbatim}
\end{frame}


\begin{frame}[fragile]
\frametitle{Example: \hol{EVEN}}
\begin{semiverbatim}\scriptsize
> \hol{val (EVEN_REL_rules, EVEN_REL_ind, EVEN_REL_cases) = Hol_reln 
  `(EVEN_REL 0) \holAnd{} (!n. EVEN_REL n ==> (EVEN_REL (n + 2)))`};

val EVEN_REL_cases =
 |- !a0. EVEN_REL a0 <=> (a0 = 0) \holOr{} ?n. (a0 = n + 2) \holAnd{} EVEN_REL n

val EVEN_REL_rules =
 |- EVEN_REL 0 \holAnd{} !n. EVEN_REL n ==> EVEN_REL (n + 2)

val EVEN_REL_ind = |- !EVEN_REL'.
   (EVEN_REL' 0 \holAnd{} (!n. EVEN_REL' n ==> EVEN_REL' (n + 2))) ==>
   (!a0. EVEN_REL a0 ==> EVEN_REL' a0)
\end{semiverbatim}
\begin{itemize}
\item notice that in this example there is exactly one fixpoint
\item therefore, for these rules the inductive and coinductive relation coincide
\end{itemize}
\end{frame}


\begin{frame}[fragile]
\frametitle{Example: Dummy Relations}
\begin{semiverbatim}\scriptsize
> \hol{val (DF_rules, DF_ind, DF_cases) = Hol_reln 
  `(!n. DF (n+1) ==> (DF n))`}

> \hol{val (DT_rules, DT_coind, DT_cases) = Hol_coreln 
  `(!n. DT (n+1) ==> (DT n))`}

val DT_coind =
  |- !DT'. (!a0. DT' a0 ==> DT' (a0 + 1)) ==> !a0. DT' a0 ==> DT a0

val DF_ind =
  |- !DF'. (!n. DF' (n + 1) ==> DF' n) ==> !a0. DF a0 ==> DF' a0

val DT_cases = |- !a0. DT a0 <=> DT (a0 + 1):
val DF_cases = |- !a0. DF a0 <=> DF (a0 + 1):
\end{semiverbatim}
\begin{itemize}
\item notice that the definitions of \hol{DT} and \hol{DF} look like a non-terminating recursive definition
\item \hol{DT} is always true, \ie \hol{|- !n.\ DT n}
\item \hol{DF} is always false, \ie \hol{|- !n.\ \holNeg{}(DF n)}
\end{itemize}
\end{frame}



\section{Quotient Types}

\begin{frame}
\frametitle{Quotient Types}

\begin{itemize}
\item \ml{quotientLib} allows to define types as quotients of existing types with respect to \emph{partial equivalence relation}
\item each equivalence class becomes a value of the new type
\item partiality allows ignoring certain values of original type
\item \ml{quotientLib} allows to lift definitions and lemmata as well
\item details are technical and won't be presented here
\end{itemize}

\end{frame}



\begin{frame}
\frametitle{Quotient Types Example}

\begin{itemize}
\item let's assume we have an implementation of finite sets of numbers as
binary trees with
\begin{itemize}
\item type \hol{binset}
\item binary tree invariant \hol{WF\_BINSET :\ binset -> bool}
\item constant \hol{empty\_binset}
\item add and member functions \hol{add :\ num -> binset -> binset},\\ \hol{mem :\ binset -> num -> bool}
\end{itemize}
\item we can define a partial equivalence relation by\\
\hol{binset\_equiv b1 b2 := (\\
\-\ \ WF\_BINSET b1 \holAnd{} WF\_BINSET b2 \holAnd{}\\
\-\ \ (!n.\ mem b1 n <=> mem b2 n))}
\item this allows defining a quotient type of sets of numbers
\item functions \hol{empty\_binset}, \hol{add} and \hol{mem} as well as lemmata about them can be lifted automatically
\end{itemize}
\end{frame}


\begin{frame}
\frametitle{Quotient Types Summary}

\begin{itemize}
\item quotient types are sometimes very useful
\begin{itemize}
\item \eg rational numbers are defined as a quotient type
\end{itemize}
\item there is powerful infrastructure for them
\item many tasks are automated
\item however, the details are technical and won't be discussed here
\end{itemize}

\end{frame}

\section{Case Expressions}

\begin{frame}
\frametitle{Pattern Matching / Case Expressions}

\begin{itemize}
\item pattern matching ubiquitous in functional programming
\item pattern matching is a powerful technique
\item it helps to write concise, readable definitions
\item very handy and frequently used for interactive theorem proving
\item however, it is \alert{not directly supported} by HOL's logic
\item representations in HOL
\begin{itemize}
\item sets of equations as produced by \hol{Define}
\item decision trees (printed as case-expressions)
\end{itemize}
\end{itemize}
\end{frame}


\begin{frame}[fragile]
\frametitle{TFL / \texttt{Define}}

\begin{itemize}
\item we have already used top-level pattern matches with the TFL package
\item \hol{Define} is able to handle them
\begin{itemize}
\item all the semantic complexity is taken care of
\item no special syntax or functions remain
\item no special rewrite rules, reasoning tools needed afterwards
\end{itemize}
\item \hol{Define} produces a set of equations
\item this is the recommended way of using pattern matching in HOL
\end{itemize}

\begin{exampleblock}{Example}
\begin{semiverbatim}\scriptsize
> \hol{val ZIP_def = Define `(ZIP (x::xs) (y::ys) = (x,y)::(ZIP xs ys)) \holAnd{}
                        (ZIP [] [] = [])`}
val ZIP_def = |- (!ys y xs x. ZIP (x::xs) (y::ys) = (x,y)::ZIP xs ys) \holAnd{}
                 (ZIP [] [] = [])
\end{semiverbatim}
\end{exampleblock}
\end{frame}


\begin{frame}[fragile]
\frametitle{Case Expressions}

\begin{itemize}
\item sometimes one does not want to use this compilation by TFL
\begin{itemize}
\item one wants to use pattern-matches somewhere nested in a term
\item one might not want to introduce a new constant
\item one might want to avoid using TFL for technical reasons
\end{itemize}
\item in such situations, case-expressions can be used
\item their syntax is similar to the syntax used by SML
\end{itemize}

\begin{exampleblock}{Example}
\begin{semiverbatim}\scriptsize
> \hol{val ZIP_def = Define `ZIP xs ys = case (xs, ys) of 
                                         (x::xs, y::ys) => (x,y)::(ZIP xs ys)
                                       | ([], []) => []`}
val ZIP_def = |- !ys xs. ZIP xs ys =
     case (xs,ys) of
       ([],[]) => []
     | ([],v4::v5) => ARB
     | (x::xs',[]) => ARB
     | (x::xs',y::ys') => (x,y)::ZIP xs' ys'
\end{semiverbatim}
\end{exampleblock}

\end{frame}



\begin{frame}[fragile]
\frametitle{Case Expressions II}

\begin{itemize}
\item the datatype package defines case-constants for each datatype
\item the parser contains a pattern compilation algorithm
\item case-expressions are by the parser compiled to decision trees using case-constants
\item pretty printer prints these decision trees as case-expressions again 
\end{itemize}

\begin{exampleblock}{Example}
\begin{semiverbatim}\scriptsize
val ZIP_def = |- !ys xs. ZIP xs ys =
     pair_CASE (xs,ys)
       (\textbsl{}v v1.
          list_CASE v (list_CASE v1 [] (\textbsl{}v4 v5. ARB))
            (\textbsl{}x xs'. list_CASE v1 ARB (\textbsl{}y ys'. (x,y)::ZIP xs' ys'))):
\end{semiverbatim}
\end{exampleblock}

\end{frame}

\begin{frame}[fragile]
\frametitle{Case Expression Issues}

\begin{itemize}
\item using case expressions feels very natural to functional programmers
\item case-expressions allow concise, well-readable definitions
\item however, there are also many drawbacks
\item there is large, complicated code in the parser and pretty printer
\begin{itemize}
\item this is outside the kernel
\item parsing a pretty-printed term can result in a non $\alpha$-equivalent one
\item there are bugs in this code (see \eg Issue \#416 reported 8 May 2017)
\end{itemize}
\item the results are hard to predict
\begin{itemize}
\item heuristics involved in creating decision tree
\item however, it is beneficial that proofs follow this internal, volatile structure
\end{itemize} 
\end{itemize}
\end{frame}


\begin{frame}[fragile]
\frametitle{Case Expression Issues II}

\begin{itemize}
\item technical issues
\begin{itemize}
\item it is tricky to reason about decision trees
\item rewrite rules about case-constants needs to be fetched from \hol{TypeBase}
\begin{itemize}
\item alternative \hol{srw\_ss} often does more than wanted
\end{itemize}
\item partially evaluated decision-trees are not pretty printed nicely any more
\end{itemize}
\item underspecified functions
\begin{itemize}
\item decision trees are exhaustive
\item they list underspecified cases explicitly with value \hol{ARB}
\item this can be lengthy
\item \hol{Define} in contrast hides underspecified cases
\end{itemize}
\end{itemize}

\end{frame}


\begin{frame}[fragile]
\frametitle{Case Expression Example I}

\begin{block}{Partial Proof Script}
\begin{semiverbatim}\scriptsize
val _ = prove (``!l1 l2. 
  (LENGTH l1 = LENGTH l2) ==> 
  ((ZIP l1 l2 = []) <=> ((l1 = []) \holAnd{} (l2 = [])))``,

ONCE_REWRITE_TAC [ZIP_def] 
\end{semiverbatim}
\end{block}


\begin{block}{Current Goal}
\begin{semiverbatim}\scriptsize
!l1 l2.
  (LENGTH l1 = LENGTH l2) ==>
  (((case (l1,l2) of
       ([],[]) => []
     | ([],v4::v5) => ARB
     | (x::xs',[]) => ARB
     | (x::xs',y::ys') => (x,y)::ZIP xs' ys') =
    []) <=> (l1 = []) \holAnd{} (l2 = []))
\end{semiverbatim}
\end{block}
\end{frame}

\begin{frame}[fragile]
\frametitle{Case Expression Example IIa -- partial evaluation}

\begin{block}{Partial Proof Script}
\begin{semiverbatim}\scriptsize
val _ = prove (``!l1 l2. 
  (LENGTH l1 = LENGTH l2) ==> 
  ((ZIP l1 l2 = []) <=> ((l1 = []) \holAnd{} (l2 = [])))``,

ONCE_REWRITE_TAC [ZIP_def] >>
REWRITE_TAC[pairTheory.pair_case_def] >> BETA_TAC
\end{semiverbatim}
\end{block}


\begin{block}{Current Goal}
\begin{semiverbatim}\scriptsize
!l1 l2.
  (LENGTH l1 = LENGTH l2) ==>
  (((case l1 of
       [] => (case l2 of [] => [] | v4::v5 => ARB)
     | x::xs' => case l2 of [] => ARB | y::ys' => (x,y)::ZIP xs' ys') =
    []) <=> (l1 = []) \holAnd{} (l2 = []))
\end{semiverbatim}
\end{block}
\end{frame}


\begin{frame}[fragile]
\frametitle{Case Expression Example IIb --- following tree structure}

\begin{block}{Partial Proof Script}
\begin{semiverbatim}\scriptsize
val _ = prove (``!l1 l2. 
  (LENGTH l1 = LENGTH l2) ==> 
  ((ZIP l1 l2 = []) <=> ((l1 = []) \holAnd{} (l2 = [])))``,

ONCE_REWRITE_TAC [ZIP_def] >>
Cases_on `l1` >| [
  REWRITE_TAC[listTheory.list_case_def] 
\end{semiverbatim}
\end{block}


\begin{block}{Current Goal}
\begin{semiverbatim}\scriptsize
!l2.
  (LENGTH [] = LENGTH l2) ==>
  (((case ([],l2) of
       ([],[]) => []
     | ([],v4::v5) => ARB
     | (x::xs',[]) => ARB
     | (x::xs',y::ys') => (x,y)::ZIP xs' ys') =
    []) <=> (l2 = []))
\end{semiverbatim}
\end{block}
\end{frame}


\begin{frame}[fragile]
\frametitle{Case Expression Summary}

\begin{itemize}
\item case expressions are natural to functional programmers
\item they allow concise, readable definitions
\item however, fancy parser and pretty-printer needed
\begin{itemize}
\item trustworthiness issues
\item sanity check lemmata advisable
\end{itemize}
\item reasoning about case expressions can be tricky and lengthy
\item proofs about case expression often hard to maintain
\item therefore, use top-level pattern matching via \hol{Define} if easily possible
\end{itemize}
\end{frame}


%%% Local Variables:
%%% mode: latex
%%% TeX-master: "current"
%%% End:
